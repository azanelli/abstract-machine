\section{Direttive}
\label{sec:direttive}
Le direttive vengono utilizzate per definire l'inizio e la fine del programma, per dichirare le variabili globali e le funzioni.

\subsection{Tabella direttive}
\label{sec:tabella_direttive}
{\footnotesize
\begin{tabularx}{\columnwidth}{l X}
\toprule
\rowcolor[gray]{0.9}
  \textbf{Direttiva} &
  \textbf{Descrizione} \\
\toprule
  \texttt{.class~public Main} &
  Inizio del programma \\
\midrule
  \texttt{.super~java/lang/Object} &
  Inizio del programma (da mettere subito dopo \texttt{.class~public Main}) \\
\midrule
  \texttt{.end~class} &
  Fine del programma \\
\midrule
  \texttt{.field public static \textit{f} \textit{d}} &
  Variabile globale \texttt{\textit{f}} \\
\midrule
  \texttt{.method public static \textit{m} \textit{d}} &
  Inizio della funzione \texttt{\textit{m}} \\
\midrule
  \texttt{.end method} &
  Fine della funzione \\
\bottomrule
\end{tabularx}
} % end \footnotesize

\subsection{Inizio e fine programma}
Un programma ha inizio dopo le due seguenti direttive
\begin{quote}
  \texttt{.class public Main} \\
  \texttt{.super java/lang/Object}
\end{quote}
e finisce con la direttiva
\begin{quote}
  \texttt{.end class}
\end{quote}
Fra queste direttive di inizio e fine del programma vengono definite le funzioni e le variabili globali. Ogni scritta prima di \texttt{.class~public Main} e dopo \texttt{.end~class} \`e considerata errore.
\subsubsection*{Esempio}
\begin{verbatim}
.class public Main
.super java/lang/Object
  ; funzioni e variabili globali
.end class
\end{verbatim}

\subsection{Variabili globali}
Le variabili globali si definiscono con la direttiva
\begin{quote}
  \texttt{.field public static \textit{fieldname} \textit{descriptor}}
\end{quote}
dove \texttt{\textit{fieldname}} \`e il nome della variabile e \texttt{\textit{descriptor}} \`e il tipo della variabile (vedere tabella dei tipi: \ref{sec:tabella_tipi}).
\subsubsection*{Esempio}
Variabile globale di nome \texttt{numberOfElements} e di tipo \texttt{int}:
\begin{verbatim}
.field public static numberOfElements I
\end{verbatim}

\subsection{Funzioni}
Per la dichiarazione di una funzione e per indicarne l'inizio si usa la direttiva
\begin{quote}
  \texttt{.method public static \textit{methodname} \textit{descriptor}}
\end{quote}
dove \texttt{\textit{methodname}} \`e il nome della funzione e \texttt{\textit{decriptor}} descrive gli argomenti e il tipo di ritorno della funzione (vedere la tabella dei tipi: \ref{tab:tabella_tipi}). \\
Per indicare la fine della definizione della funzione si usa la direttiva
\begin{quote}
  \texttt{.end method}
\end{quote}
Fra le direttive \texttt{.method} e \texttt{.end~method} vi sono le istruzioni della funzione (vedere le tabelle delle istruzioni nel paragrafo \ref{sec:tabelle_istruzioni}).
\subsubsection*{Esempio}
Funzione di nome \texttt{icalc} che prende come argomenti quattro interi e ritorna un intero
\begin{verbatim}
.method public static icalc(IIII)I
  ; serie di istruzioni
.end method
\end{verbatim}

\subsection{Funzione inizializzatore}
Se esiste \`e la prima funzione ad essere eseguita ed \`e utilizzata per inizializzare le variabili globali prima di eseguire la funzione principale; ha nome \texttt{<clinit>}, non prende argomenti e ha tipo di ritorno void:
\begin{quote}
  \texttt{.method public static <clinit> ()V}
\end{quote}
Se nel testo del programma non c'\`e la funzione \texttt{<clinit>} viene eseguita direttamente la funzione principale.
\subsubsection*{Esempio}
\begin{verbatim}
.method public static <clinit> ()V
  ; serie di istruzioni per inizializzare
  ; le variabili globali
  return
.end method
\end{verbatim}

\subsection{Funzione principale}
\`E la funzione principale, la prima ad essere eseguita (se non esiste la funzione inizializzatore) e da cui inizia il programma, ha nome \texttt{main}:
\begin{quote}
  \texttt{.method public static main([Ljava/lang/String;)V}
\end{quote}
Se nel testo del programma non c'\`e la funzione \texttt{main} viene segnalato un errore e il programma non pu\`o essere eseguito.
\subsubsection*{Esempio}
\begin{verbatim}
.method public static main([Ljava/lang/String;)V
  return
.end method
\end{verbatim}
