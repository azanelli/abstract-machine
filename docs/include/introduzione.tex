\section{Introduzione}
\label{sec:introduzione}
In questo documento viene elencato e descritto l'instruction set della macchina astratta e viene illustrata la sua struttura di funzionamento. In entrambi i casi \`e basata sulla Java Virtual Machine (JVM). Per l'instruction set \`e stato preso un sottoinsieme di quello della JVM, percui un programma funzionante su questa macchina astratta funziona anche sulla JVM, traducendo per\`o gli mnemonici in bytecode con un assembler come, per esempio, l'Oolong scritto da Joshua Engel. La struttura, in modo simile alla JVM, \`e composta da uno stack di sistema, in cui vengono messi i record di attivazione (RdA) di ogni funzione che viene eseguita, e da una area-programma dove vengono memorizzate le istruzioni da eseguire. Ogni RdA \`e quindi composto da un program counter (PC) che punta all'istruzione da eseguire, da uno stack degli operandi e da un array delle variabili locali.
